\documentclass[ngerman]{article}
\usepackage{babel}
\usepackage{siunitx}

\begin{document}

\section{Allgemeines}

Geschwindigtkeit 2.3 Meter pro Sekunde:
\[v=\SI{2.3}{\metre\per\second}\]

Formel:

\[
v=\frac{s}{t}
\]


Beispiel:

Fahrzeug fährt eine Strecke von 500m in 25 sek.

\[
s=500m
\]


\[
t=25s
\]


Lösung:

\[
v=\frac{500m}{25s}=\frac{20m}{s}=20*3.6\frac{kms}{mh}=72km/h
\]


\section{Beispiel zu 'Ohmsches Gesetz'}
In einem Stromkreis fliessen \SI{100}{\milli\ampere} Strom. Die Spannungsquelle hat
\SI{9}{\volt}. Wie groß ist der Gesamtwiderstand im Stromkreis?

\[
R = \frac{U}{I}
\]

\[
R = \frac{\SI{9}{\volt}}{\SI{100}{\milli\ampere}} = \SI{90}{\Omega}
\]

\[
\SI{25}{\Omega} \==
\SI{25}{\volt\per\ampere}
\]

\end{document}
